
% Default to the notebook output style

    


% Inherit from the specified cell style.




    
\documentclass[11pt]{article}

    
    
    \usepackage[T1]{fontenc}
    % Nicer default font (+ math font) than Computer Modern for most use cases
    \usepackage{mathpazo}

    % Basic figure setup, for now with no caption control since it's done
    % automatically by Pandoc (which extracts ![](path) syntax from Markdown).
    \usepackage{graphicx}
    % We will generate all images so they have a width \maxwidth. This means
    % that they will get their normal width if they fit onto the page, but
    % are scaled down if they would overflow the margins.
    \makeatletter
    \def\maxwidth{\ifdim\Gin@nat@width>\linewidth\linewidth
    \else\Gin@nat@width\fi}
    \makeatother
    \let\Oldincludegraphics\includegraphics
    % Set max figure width to be 80% of text width, for now hardcoded.
    \renewcommand{\includegraphics}[1]{\Oldincludegraphics[width=.8\maxwidth]{#1}}
    % Ensure that by default, figures have no caption (until we provide a
    % proper Figure object with a Caption API and a way to capture that
    % in the conversion process - todo).
    \usepackage{caption}
    \DeclareCaptionLabelFormat{nolabel}{}
    \captionsetup{labelformat=nolabel}

    \usepackage{adjustbox} % Used to constrain images to a maximum size 
    \usepackage{xcolor} % Allow colors to be defined
    \usepackage{enumerate} % Needed for markdown enumerations to work
    \usepackage{geometry} % Used to adjust the document margins
    \usepackage{amsmath} % Equations
    \usepackage{amssymb} % Equations
    \usepackage{textcomp} % defines textquotesingle
    % Hack from http://tex.stackexchange.com/a/47451/13684:
    \AtBeginDocument{%
        \def\PYZsq{\textquotesingle}% Upright quotes in Pygmentized code
    }
    \usepackage{upquote} % Upright quotes for verbatim code
    \usepackage{eurosym} % defines \euro
    \usepackage[mathletters]{ucs} % Extended unicode (utf-8) support
    \usepackage[utf8x]{inputenc} % Allow utf-8 characters in the tex document
    \usepackage{fancyvrb} % verbatim replacement that allows latex
    \usepackage{grffile} % extends the file name processing of package graphics 
                         % to support a larger range 
    % The hyperref package gives us a pdf with properly built
    % internal navigation ('pdf bookmarks' for the table of contents,
    % internal cross-reference links, web links for URLs, etc.)
    \usepackage{hyperref}
    \usepackage{longtable} % longtable support required by pandoc >1.10
    \usepackage{booktabs}  % table support for pandoc > 1.12.2
    \usepackage[inline]{enumitem} % IRkernel/repr support (it uses the enumerate* environment)
    \usepackage[normalem]{ulem} % ulem is needed to support strikethroughs (\sout)
                                % normalem makes italics be italics, not underlines
    

    
    
    % Colors for the hyperref package
    \definecolor{urlcolor}{rgb}{0,.145,.698}
    \definecolor{linkcolor}{rgb}{.71,0.21,0.01}
    \definecolor{citecolor}{rgb}{.12,.54,.11}

    % ANSI colors
    \definecolor{ansi-black}{HTML}{3E424D}
    \definecolor{ansi-black-intense}{HTML}{282C36}
    \definecolor{ansi-red}{HTML}{E75C58}
    \definecolor{ansi-red-intense}{HTML}{B22B31}
    \definecolor{ansi-green}{HTML}{00A250}
    \definecolor{ansi-green-intense}{HTML}{007427}
    \definecolor{ansi-yellow}{HTML}{DDB62B}
    \definecolor{ansi-yellow-intense}{HTML}{B27D12}
    \definecolor{ansi-blue}{HTML}{208FFB}
    \definecolor{ansi-blue-intense}{HTML}{0065CA}
    \definecolor{ansi-magenta}{HTML}{D160C4}
    \definecolor{ansi-magenta-intense}{HTML}{A03196}
    \definecolor{ansi-cyan}{HTML}{60C6C8}
    \definecolor{ansi-cyan-intense}{HTML}{258F8F}
    \definecolor{ansi-white}{HTML}{C5C1B4}
    \definecolor{ansi-white-intense}{HTML}{A1A6B2}

    % commands and environments needed by pandoc snippets
    % extracted from the output of `pandoc -s`
    \providecommand{\tightlist}{%
      \setlength{\itemsep}{0pt}\setlength{\parskip}{0pt}}
    \DefineVerbatimEnvironment{Highlighting}{Verbatim}{commandchars=\\\{\}}
    % Add ',fontsize=\small' for more characters per line
    \newenvironment{Shaded}{}{}
    \newcommand{\KeywordTok}[1]{\textcolor[rgb]{0.00,0.44,0.13}{\textbf{{#1}}}}
    \newcommand{\DataTypeTok}[1]{\textcolor[rgb]{0.56,0.13,0.00}{{#1}}}
    \newcommand{\DecValTok}[1]{\textcolor[rgb]{0.25,0.63,0.44}{{#1}}}
    \newcommand{\BaseNTok}[1]{\textcolor[rgb]{0.25,0.63,0.44}{{#1}}}
    \newcommand{\FloatTok}[1]{\textcolor[rgb]{0.25,0.63,0.44}{{#1}}}
    \newcommand{\CharTok}[1]{\textcolor[rgb]{0.25,0.44,0.63}{{#1}}}
    \newcommand{\StringTok}[1]{\textcolor[rgb]{0.25,0.44,0.63}{{#1}}}
    \newcommand{\CommentTok}[1]{\textcolor[rgb]{0.38,0.63,0.69}{\textit{{#1}}}}
    \newcommand{\OtherTok}[1]{\textcolor[rgb]{0.00,0.44,0.13}{{#1}}}
    \newcommand{\AlertTok}[1]{\textcolor[rgb]{1.00,0.00,0.00}{\textbf{{#1}}}}
    \newcommand{\FunctionTok}[1]{\textcolor[rgb]{0.02,0.16,0.49}{{#1}}}
    \newcommand{\RegionMarkerTok}[1]{{#1}}
    \newcommand{\ErrorTok}[1]{\textcolor[rgb]{1.00,0.00,0.00}{\textbf{{#1}}}}
    \newcommand{\NormalTok}[1]{{#1}}
    
    % Additional commands for more recent versions of Pandoc
    \newcommand{\ConstantTok}[1]{\textcolor[rgb]{0.53,0.00,0.00}{{#1}}}
    \newcommand{\SpecialCharTok}[1]{\textcolor[rgb]{0.25,0.44,0.63}{{#1}}}
    \newcommand{\VerbatimStringTok}[1]{\textcolor[rgb]{0.25,0.44,0.63}{{#1}}}
    \newcommand{\SpecialStringTok}[1]{\textcolor[rgb]{0.73,0.40,0.53}{{#1}}}
    \newcommand{\ImportTok}[1]{{#1}}
    \newcommand{\DocumentationTok}[1]{\textcolor[rgb]{0.73,0.13,0.13}{\textit{{#1}}}}
    \newcommand{\AnnotationTok}[1]{\textcolor[rgb]{0.38,0.63,0.69}{\textbf{\textit{{#1}}}}}
    \newcommand{\CommentVarTok}[1]{\textcolor[rgb]{0.38,0.63,0.69}{\textbf{\textit{{#1}}}}}
    \newcommand{\VariableTok}[1]{\textcolor[rgb]{0.10,0.09,0.49}{{#1}}}
    \newcommand{\ControlFlowTok}[1]{\textcolor[rgb]{0.00,0.44,0.13}{\textbf{{#1}}}}
    \newcommand{\OperatorTok}[1]{\textcolor[rgb]{0.40,0.40,0.40}{{#1}}}
    \newcommand{\BuiltInTok}[1]{{#1}}
    \newcommand{\ExtensionTok}[1]{{#1}}
    \newcommand{\PreprocessorTok}[1]{\textcolor[rgb]{0.74,0.48,0.00}{{#1}}}
    \newcommand{\AttributeTok}[1]{\textcolor[rgb]{0.49,0.56,0.16}{{#1}}}
    \newcommand{\InformationTok}[1]{\textcolor[rgb]{0.38,0.63,0.69}{\textbf{\textit{{#1}}}}}
    \newcommand{\WarningTok}[1]{\textcolor[rgb]{0.38,0.63,0.69}{\textbf{\textit{{#1}}}}}
    
    
    % Define a nice break command that doesn't care if a line doesn't already
    % exist.
    \def\br{\hspace*{\fill} \\* }
    % Math Jax compatability definitions
    \def\gt{>}
    \def\lt{<}
    % Document parameters
    \title{HW2}
    
    
    

    % Pygments definitions
    
\makeatletter
\def\PY@reset{\let\PY@it=\relax \let\PY@bf=\relax%
    \let\PY@ul=\relax \let\PY@tc=\relax%
    \let\PY@bc=\relax \let\PY@ff=\relax}
\def\PY@tok#1{\csname PY@tok@#1\endcsname}
\def\PY@toks#1+{\ifx\relax#1\empty\else%
    \PY@tok{#1}\expandafter\PY@toks\fi}
\def\PY@do#1{\PY@bc{\PY@tc{\PY@ul{%
    \PY@it{\PY@bf{\PY@ff{#1}}}}}}}
\def\PY#1#2{\PY@reset\PY@toks#1+\relax+\PY@do{#2}}

\expandafter\def\csname PY@tok@w\endcsname{\def\PY@tc##1{\textcolor[rgb]{0.73,0.73,0.73}{##1}}}
\expandafter\def\csname PY@tok@c\endcsname{\let\PY@it=\textit\def\PY@tc##1{\textcolor[rgb]{0.25,0.50,0.50}{##1}}}
\expandafter\def\csname PY@tok@cp\endcsname{\def\PY@tc##1{\textcolor[rgb]{0.74,0.48,0.00}{##1}}}
\expandafter\def\csname PY@tok@k\endcsname{\let\PY@bf=\textbf\def\PY@tc##1{\textcolor[rgb]{0.00,0.50,0.00}{##1}}}
\expandafter\def\csname PY@tok@kp\endcsname{\def\PY@tc##1{\textcolor[rgb]{0.00,0.50,0.00}{##1}}}
\expandafter\def\csname PY@tok@kt\endcsname{\def\PY@tc##1{\textcolor[rgb]{0.69,0.00,0.25}{##1}}}
\expandafter\def\csname PY@tok@o\endcsname{\def\PY@tc##1{\textcolor[rgb]{0.40,0.40,0.40}{##1}}}
\expandafter\def\csname PY@tok@ow\endcsname{\let\PY@bf=\textbf\def\PY@tc##1{\textcolor[rgb]{0.67,0.13,1.00}{##1}}}
\expandafter\def\csname PY@tok@nb\endcsname{\def\PY@tc##1{\textcolor[rgb]{0.00,0.50,0.00}{##1}}}
\expandafter\def\csname PY@tok@nf\endcsname{\def\PY@tc##1{\textcolor[rgb]{0.00,0.00,1.00}{##1}}}
\expandafter\def\csname PY@tok@nc\endcsname{\let\PY@bf=\textbf\def\PY@tc##1{\textcolor[rgb]{0.00,0.00,1.00}{##1}}}
\expandafter\def\csname PY@tok@nn\endcsname{\let\PY@bf=\textbf\def\PY@tc##1{\textcolor[rgb]{0.00,0.00,1.00}{##1}}}
\expandafter\def\csname PY@tok@ne\endcsname{\let\PY@bf=\textbf\def\PY@tc##1{\textcolor[rgb]{0.82,0.25,0.23}{##1}}}
\expandafter\def\csname PY@tok@nv\endcsname{\def\PY@tc##1{\textcolor[rgb]{0.10,0.09,0.49}{##1}}}
\expandafter\def\csname PY@tok@no\endcsname{\def\PY@tc##1{\textcolor[rgb]{0.53,0.00,0.00}{##1}}}
\expandafter\def\csname PY@tok@nl\endcsname{\def\PY@tc##1{\textcolor[rgb]{0.63,0.63,0.00}{##1}}}
\expandafter\def\csname PY@tok@ni\endcsname{\let\PY@bf=\textbf\def\PY@tc##1{\textcolor[rgb]{0.60,0.60,0.60}{##1}}}
\expandafter\def\csname PY@tok@na\endcsname{\def\PY@tc##1{\textcolor[rgb]{0.49,0.56,0.16}{##1}}}
\expandafter\def\csname PY@tok@nt\endcsname{\let\PY@bf=\textbf\def\PY@tc##1{\textcolor[rgb]{0.00,0.50,0.00}{##1}}}
\expandafter\def\csname PY@tok@nd\endcsname{\def\PY@tc##1{\textcolor[rgb]{0.67,0.13,1.00}{##1}}}
\expandafter\def\csname PY@tok@s\endcsname{\def\PY@tc##1{\textcolor[rgb]{0.73,0.13,0.13}{##1}}}
\expandafter\def\csname PY@tok@sd\endcsname{\let\PY@it=\textit\def\PY@tc##1{\textcolor[rgb]{0.73,0.13,0.13}{##1}}}
\expandafter\def\csname PY@tok@si\endcsname{\let\PY@bf=\textbf\def\PY@tc##1{\textcolor[rgb]{0.73,0.40,0.53}{##1}}}
\expandafter\def\csname PY@tok@se\endcsname{\let\PY@bf=\textbf\def\PY@tc##1{\textcolor[rgb]{0.73,0.40,0.13}{##1}}}
\expandafter\def\csname PY@tok@sr\endcsname{\def\PY@tc##1{\textcolor[rgb]{0.73,0.40,0.53}{##1}}}
\expandafter\def\csname PY@tok@ss\endcsname{\def\PY@tc##1{\textcolor[rgb]{0.10,0.09,0.49}{##1}}}
\expandafter\def\csname PY@tok@sx\endcsname{\def\PY@tc##1{\textcolor[rgb]{0.00,0.50,0.00}{##1}}}
\expandafter\def\csname PY@tok@m\endcsname{\def\PY@tc##1{\textcolor[rgb]{0.40,0.40,0.40}{##1}}}
\expandafter\def\csname PY@tok@gh\endcsname{\let\PY@bf=\textbf\def\PY@tc##1{\textcolor[rgb]{0.00,0.00,0.50}{##1}}}
\expandafter\def\csname PY@tok@gu\endcsname{\let\PY@bf=\textbf\def\PY@tc##1{\textcolor[rgb]{0.50,0.00,0.50}{##1}}}
\expandafter\def\csname PY@tok@gd\endcsname{\def\PY@tc##1{\textcolor[rgb]{0.63,0.00,0.00}{##1}}}
\expandafter\def\csname PY@tok@gi\endcsname{\def\PY@tc##1{\textcolor[rgb]{0.00,0.63,0.00}{##1}}}
\expandafter\def\csname PY@tok@gr\endcsname{\def\PY@tc##1{\textcolor[rgb]{1.00,0.00,0.00}{##1}}}
\expandafter\def\csname PY@tok@ge\endcsname{\let\PY@it=\textit}
\expandafter\def\csname PY@tok@gs\endcsname{\let\PY@bf=\textbf}
\expandafter\def\csname PY@tok@gp\endcsname{\let\PY@bf=\textbf\def\PY@tc##1{\textcolor[rgb]{0.00,0.00,0.50}{##1}}}
\expandafter\def\csname PY@tok@go\endcsname{\def\PY@tc##1{\textcolor[rgb]{0.53,0.53,0.53}{##1}}}
\expandafter\def\csname PY@tok@gt\endcsname{\def\PY@tc##1{\textcolor[rgb]{0.00,0.27,0.87}{##1}}}
\expandafter\def\csname PY@tok@err\endcsname{\def\PY@bc##1{\setlength{\fboxsep}{0pt}\fcolorbox[rgb]{1.00,0.00,0.00}{1,1,1}{\strut ##1}}}
\expandafter\def\csname PY@tok@kc\endcsname{\let\PY@bf=\textbf\def\PY@tc##1{\textcolor[rgb]{0.00,0.50,0.00}{##1}}}
\expandafter\def\csname PY@tok@kd\endcsname{\let\PY@bf=\textbf\def\PY@tc##1{\textcolor[rgb]{0.00,0.50,0.00}{##1}}}
\expandafter\def\csname PY@tok@kn\endcsname{\let\PY@bf=\textbf\def\PY@tc##1{\textcolor[rgb]{0.00,0.50,0.00}{##1}}}
\expandafter\def\csname PY@tok@kr\endcsname{\let\PY@bf=\textbf\def\PY@tc##1{\textcolor[rgb]{0.00,0.50,0.00}{##1}}}
\expandafter\def\csname PY@tok@bp\endcsname{\def\PY@tc##1{\textcolor[rgb]{0.00,0.50,0.00}{##1}}}
\expandafter\def\csname PY@tok@fm\endcsname{\def\PY@tc##1{\textcolor[rgb]{0.00,0.00,1.00}{##1}}}
\expandafter\def\csname PY@tok@vc\endcsname{\def\PY@tc##1{\textcolor[rgb]{0.10,0.09,0.49}{##1}}}
\expandafter\def\csname PY@tok@vg\endcsname{\def\PY@tc##1{\textcolor[rgb]{0.10,0.09,0.49}{##1}}}
\expandafter\def\csname PY@tok@vi\endcsname{\def\PY@tc##1{\textcolor[rgb]{0.10,0.09,0.49}{##1}}}
\expandafter\def\csname PY@tok@vm\endcsname{\def\PY@tc##1{\textcolor[rgb]{0.10,0.09,0.49}{##1}}}
\expandafter\def\csname PY@tok@sa\endcsname{\def\PY@tc##1{\textcolor[rgb]{0.73,0.13,0.13}{##1}}}
\expandafter\def\csname PY@tok@sb\endcsname{\def\PY@tc##1{\textcolor[rgb]{0.73,0.13,0.13}{##1}}}
\expandafter\def\csname PY@tok@sc\endcsname{\def\PY@tc##1{\textcolor[rgb]{0.73,0.13,0.13}{##1}}}
\expandafter\def\csname PY@tok@dl\endcsname{\def\PY@tc##1{\textcolor[rgb]{0.73,0.13,0.13}{##1}}}
\expandafter\def\csname PY@tok@s2\endcsname{\def\PY@tc##1{\textcolor[rgb]{0.73,0.13,0.13}{##1}}}
\expandafter\def\csname PY@tok@sh\endcsname{\def\PY@tc##1{\textcolor[rgb]{0.73,0.13,0.13}{##1}}}
\expandafter\def\csname PY@tok@s1\endcsname{\def\PY@tc##1{\textcolor[rgb]{0.73,0.13,0.13}{##1}}}
\expandafter\def\csname PY@tok@mb\endcsname{\def\PY@tc##1{\textcolor[rgb]{0.40,0.40,0.40}{##1}}}
\expandafter\def\csname PY@tok@mf\endcsname{\def\PY@tc##1{\textcolor[rgb]{0.40,0.40,0.40}{##1}}}
\expandafter\def\csname PY@tok@mh\endcsname{\def\PY@tc##1{\textcolor[rgb]{0.40,0.40,0.40}{##1}}}
\expandafter\def\csname PY@tok@mi\endcsname{\def\PY@tc##1{\textcolor[rgb]{0.40,0.40,0.40}{##1}}}
\expandafter\def\csname PY@tok@il\endcsname{\def\PY@tc##1{\textcolor[rgb]{0.40,0.40,0.40}{##1}}}
\expandafter\def\csname PY@tok@mo\endcsname{\def\PY@tc##1{\textcolor[rgb]{0.40,0.40,0.40}{##1}}}
\expandafter\def\csname PY@tok@ch\endcsname{\let\PY@it=\textit\def\PY@tc##1{\textcolor[rgb]{0.25,0.50,0.50}{##1}}}
\expandafter\def\csname PY@tok@cm\endcsname{\let\PY@it=\textit\def\PY@tc##1{\textcolor[rgb]{0.25,0.50,0.50}{##1}}}
\expandafter\def\csname PY@tok@cpf\endcsname{\let\PY@it=\textit\def\PY@tc##1{\textcolor[rgb]{0.25,0.50,0.50}{##1}}}
\expandafter\def\csname PY@tok@c1\endcsname{\let\PY@it=\textit\def\PY@tc##1{\textcolor[rgb]{0.25,0.50,0.50}{##1}}}
\expandafter\def\csname PY@tok@cs\endcsname{\let\PY@it=\textit\def\PY@tc##1{\textcolor[rgb]{0.25,0.50,0.50}{##1}}}

\def\PYZbs{\char`\\}
\def\PYZus{\char`\_}
\def\PYZob{\char`\{}
\def\PYZcb{\char`\}}
\def\PYZca{\char`\^}
\def\PYZam{\char`\&}
\def\PYZlt{\char`\<}
\def\PYZgt{\char`\>}
\def\PYZsh{\char`\#}
\def\PYZpc{\char`\%}
\def\PYZdl{\char`\$}
\def\PYZhy{\char`\-}
\def\PYZsq{\char`\'}
\def\PYZdq{\char`\"}
\def\PYZti{\char`\~}
% for compatibility with earlier versions
\def\PYZat{@}
\def\PYZlb{[}
\def\PYZrb{]}
\makeatother


    % Exact colors from NB
    \definecolor{incolor}{rgb}{0.0, 0.0, 0.5}
    \definecolor{outcolor}{rgb}{0.545, 0.0, 0.0}



    
    % Prevent overflowing lines due to hard-to-break entities
    \sloppy 
    % Setup hyperref package
    \hypersetup{
      breaklinks=true,  % so long urls are correctly broken across lines
      colorlinks=true,
      urlcolor=urlcolor,
      linkcolor=linkcolor,
      citecolor=citecolor,
      }
    % Slightly bigger margins than the latex defaults
    
    \geometry{verbose,tmargin=1in,bmargin=1in,lmargin=1in,rmargin=1in}
    
    

    \begin{document}
    
    
    \maketitle
    
    

    
    \section{Homework 2}\label{homework-2}

\subsection{Due Thursday, September 20th 2018 at 11:59
PM.}\label{due-thursday-september-20th-2018-at-1159-pm.}

\subsubsection{Be sure to push the final version of your notebook to
your GitHub repo. Follow the instructions on the course
website.}\label{be-sure-to-push-the-final-version-of-your-notebook-to-your-github-repo.-follow-the-instructions-on-the-course-website.}

\subsubsection{Topics}\label{topics}

\paragraph{\texorpdfstring{Section \ref{part_1}: Get Familiar with
\texttt{git}}{: Get Familiar with git}}\label{part-1-get-familiar-with-git}

\begin{itemize}
\tightlist
\item
  Section \ref{p1}. Git and recovering from a mistake
\item
  Section \ref{p2}. Git and checking out a single file
\end{itemize}

\paragraph{\texorpdfstring{Section \ref{part_2}: Python
Basics}{: Python Basics}}\label{part-2-python-basics}

\begin{itemize}
\tightlist
\item
  Section \ref{p3}. Count and plot
\item
  Section \ref{p4}. Average area of the circles (part 1)
\end{itemize}

\paragraph{\texorpdfstring{Section \ref{part_3}: Closure and
Decorators}{: Closure and Decorators}}\label{part-3-closure-and-decorators}

\begin{itemize}
\tightlist
\item
  Section \ref{p5}. Simple bank account withdraw system
\item
  Section \ref{p6}. Average area of the circles (part 2)
\item
  Section \ref{p7}. Positivity
\end{itemize}

\begin{center}\rule{0.5\linewidth}{\linethickness}\end{center}

    \begin{center}\rule{0.5\linewidth}{\linethickness}\end{center}

 \#\# Part 1 {[}10 pts{]}: Get Familiar with GitHub \textbf{Note:} Start
this problem after Lecture 3.

 \#\#\# Problem 1 {[}5 pts{]}: \texttt{Git} and recovering from a
mistake Here's the scenario for this problem. In Lecture, you created a
branch in your \texttt{playground} repo called \texttt{mybranch1}. You
worked on \texttt{mybranch1} and created a file called
\texttt{books.md}, but you never merged that file into your
\texttt{master} branch. In fact, we'll operate under the assumption that
you really \emph{don't} want to merge \texttt{books.md} into
\texttt{master} yet. However, being human, you are bound to make
mistakes. In this problem, you will clone \texttt{playground} and forget
to switch to \texttt{mybranch1}. You will accidentally pull
\texttt{books.md} into \texttt{master}. Then, you will recover from this
mistake by reverting \texttt{master} to an earlier state; one that
doesn't have \texttt{books.md}.

You will do this problem in the \texttt{Jupyter} notebook so we can see
your output. Once again, you will work with your \texttt{playground}
repository (created and explored in Lectures 2 and 3).

\textbf{NOTE:} At the beginning of each \texttt{Jupyter} notebook cell,
you MUST type \texttt{\%\%bash}. If you don't do that then you will not
be able to work with the necessary bash commands.

Follow the following steps for this problem:

\begin{enumerate}
\def\labelenumi{\arabic{enumi}.}
\tightlist
\item
  First cell:
\item
  Type \texttt{cd\ /tmp} to enter the temporary directory
\item
  \texttt{git\ clone\ \textless{}url\ to\ your\ playground\ repo\textgreater{}}
  \textbf{Note:} If you want to re-clone the repo (after you already
  cloned it), make sure you delete the old repo using
  \texttt{rm\ -r\ /tmp/playground}.
\item
  Second cell:
\item
  Go into your local \texttt{playground} directory
  (\texttt{cd\ /tmp/playground})
\item
  Type \texttt{git\ pull\ origin\ mybranch1}
\item
  \texttt{ls} Uh oh! \texttt{books.md} is now in your \texttt{master}
  branch!
\item
  Third cell:
\item
  Go into your local \texttt{playground} directory
  (\texttt{cd\ /tmp/playground})
\item
  Type \texttt{git\ status} Your local \texttt{master} branch is now
  ahead of the remote \texttt{master}!
\item
  Fourth cell:
\item
  Go into your local \texttt{playground} directory
  (\texttt{cd\ /tmp/playground})
\item
  Type \texttt{git\ reset\ -\/-hard\ origin/master}
\item
  \texttt{ls} \texttt{books.md} is gone!
\item
  Fifth cell:
\item
  Go into your local \texttt{playground} directory
  (\texttt{cd\ /tmp/playground})
\item
  Type \texttt{git\ status} Everything is back to normal.
\end{enumerate}

The whole point of this problem was to show you how to get your local
repo back to an earlier state. In this exercise, you accidentally merged
something to \texttt{master} that you didn't want. Rather than starting
to delete things all over the place, you can simply reset your
\texttt{HEAD} to a previous commit.

\textbf{IMPORTANT!} This is a proper \texttt{git} workflow. DO NOT start
deleting, rebasing, or re-cloning everything. Take one step at a time,
think about the decisions you're making, and execute deliberately. It
will take a little practice in the beginning, but time spent now will
translate to lots of time saved later.

    \begin{Verbatim}[commandchars=\\\{\}]
{\color{incolor}In [{\color{incolor}1}]:} \PYZpc{}\PYZpc{}bash
        \PY{n+nb}{cd} /tmp
        rm \PYZhy{}rf playground \PY{c+c1}{\PYZsh{}remove if it exists}
        git clone https://github.com/cheunjm/playground.git
\end{Verbatim}


    \begin{Verbatim}[commandchars=\\\{\}]
Cloning into 'playground'{\ldots}

    \end{Verbatim}

    \begin{Verbatim}[commandchars=\\\{\}]
{\color{incolor}In [{\color{incolor}2}]:} \PYZpc{}\PYZpc{}bash
        \PY{n+nb}{cd} /tmp/playground
        git pull origin mybranch1
        ls
\end{Verbatim}


    \begin{Verbatim}[commandchars=\\\{\}]
Merge made by the 'recursive' strategy.
 books.md | 1 +
 1 file changed, 1 insertion(+)
 create mode 100644 books.md
README.md
books.md
feature.txt
messages.md
world.md

    \end{Verbatim}

    \begin{Verbatim}[commandchars=\\\{\}]
From https://github.com/cheunjm/playground
 * branch            mybranch1  -> FETCH\_HEAD

    \end{Verbatim}

    \begin{Verbatim}[commandchars=\\\{\}]
{\color{incolor}In [{\color{incolor}3}]:} \PYZpc{}\PYZpc{}bash
        \PY{n+nb}{cd} /tmp/playground
        git status
\end{Verbatim}


    \begin{Verbatim}[commandchars=\\\{\}]
On branch master
Your branch is ahead of 'origin/master' by 2 commits.
  (use "git push" to publish your local commits)
nothing to commit, working tree clean

    \end{Verbatim}

    \begin{Verbatim}[commandchars=\\\{\}]
{\color{incolor}In [{\color{incolor}4}]:} \PYZpc{}\PYZpc{}bash
        \PY{n+nb}{cd} /tmp/playground
        git reset \PYZhy{}\PYZhy{}hard origin/master
        ls
\end{Verbatim}


    \begin{Verbatim}[commandchars=\\\{\}]
HEAD is now at 4c4e6ea George finished his books!
README.md
feature.txt
messages.md
world.md

    \end{Verbatim}

    \begin{Verbatim}[commandchars=\\\{\}]
{\color{incolor}In [{\color{incolor}5}]:} \PYZpc{}\PYZpc{}bash
        \PY{n+nb}{cd} /tmp/playground
        git status
\end{Verbatim}


    \begin{Verbatim}[commandchars=\\\{\}]
On branch master
Your branch is up-to-date with 'origin/master'.
nothing to commit, working tree clean

    \end{Verbatim}

    \begin{center}\rule{0.5\linewidth}{\linethickness}\end{center}

 \#\#\# Problem 2 {[}5 pts{]}: \texttt{Git} and checking out a single
file Sometimes you don't want to merge an entire branch from the
upstream but just one file from it. There is a direct use case for such
a situation. Suppose I've made an error in this homework (or a lecture)
and want to correct it. I fix the mistake in the upstream repo. In the
meantime you have edited some other files and you really don't want to
manually ignore my older copies of those files. Rather, you want to fix
just one file from this new branch. This is how you do it.

As usual, be sure to type in \texttt{\%\%bash} before you write any
\texttt{bash} commands in a cell.

\textbf{Note:} The steps below assume that you have already cloned the
\texttt{playground} repo in this notebook.

\begin{enumerate}
\def\labelenumi{\arabic{enumi}.}
\tightlist
\item
  First cell:
\item
  Go into the \texttt{playground} repo and fetch the changes from the
  master branch of the \texttt{course} remote.
\item
  Second cell:
\item
  \texttt{git\ checkout\ course/master\ -\/-\ README.md}. The
  \texttt{-\/-} means that \texttt{README.md} is a file (as opposed to a
  \texttt{branch}).
\item
  \texttt{cat\ README.md}. This just looks at the updated file.
\item
  Third cell:
\item
  \texttt{git\ status}
\item
  Commit the changes to your local repo with an appropriate commit
  message.
\item
  \texttt{git\ status}
\item
  Push the changes to your remote repo.
\end{enumerate}

    \begin{Verbatim}[commandchars=\\\{\}]
{\color{incolor}In [{\color{incolor}8}]:} \PYZpc{}\PYZpc{}bash
        \PY{n+nb}{cd} /tmp/playground
        git remote add course https://github.com/IACS\PYZhy{}CS\PYZhy{}207/playground.git
        git fetch course
\end{Verbatim}


    \begin{Verbatim}[commandchars=\\\{\}]
From https://github.com/IACS-CS-207/playground
 * [new branch]      master     -> course/master

    \end{Verbatim}

    \begin{Verbatim}[commandchars=\\\{\}]
{\color{incolor}In [{\color{incolor}9}]:} \PYZpc{}\PYZpc{}bash
        \PY{n+nb}{cd} /tmp/playground
        git checkout course/master \PYZhy{}\PYZhy{} README.md
        cat README.md
\end{Verbatim}


    \begin{Verbatim}[commandchars=\\\{\}]
\# Playground Repo

Fall 2018 message.

    \end{Verbatim}

    \begin{Verbatim}[commandchars=\\\{\}]
{\color{incolor}In [{\color{incolor}11}]:} \PYZpc{}\PYZpc{}bash
         \PY{n+nb}{cd} /tmp/playground
         git status
         git commit \PYZhy{}m \PY{l+s+s2}{\PYZdq{}Update README\PYZdq{}}
         git push
\end{Verbatim}


    \begin{Verbatim}[commandchars=\\\{\}]
On branch master
Your branch is up-to-date with 'origin/master'.
Changes to be committed:
  (use "git reset HEAD <file>{\ldots}" to unstage)

	modified:   README.md

[master f374b23] Update README
 1 file changed, 1 insertion(+), 1 deletion(-)

    \end{Verbatim}

    \begin{Verbatim}[commandchars=\\\{\}]
To https://github.com/cheunjm/playground.git
   4c4e6ea..f374b23  master -> master

    \end{Verbatim}

    \begin{center}\rule{0.5\linewidth}{\linethickness}\end{center}

 \#\# Part 2 {[}20 pts{]}: Python Basics \#\#\# Problem 3 {[}10 pts{]}:
Count and Plot In this problem, you will make a bar plot of the computer
languages that students who take CS207 know. The file
\texttt{languages.txt} contains all the languages that students listed
as their primary language in the course survey from a previous iteration
of the course.

Do the following: * Load the language strings from the file into a list.
* Use the \texttt{Counter} method from the \texttt{collections} library
to count the number of occurrences of each element of the list.
\textbf{Print out your answer to the screen.} * Create a bar plot to
display the frequency of each language. Be sure to label the x-axis! +
Remember, to create plots in the notebook you must put the line
\texttt{\%matplotlib\ inline} at the beginning of your notebook. You may
also want to check out \texttt{\%matplotlib\ notebook}. + Be sure to
import matplotlib: \texttt{import\ matplotlib.pyplot\ as\ plt}. + To
generate the bar plot write \texttt{plt.bar(x\_coords,\ freqs)}. You
need to define \texttt{x\_coords} and \texttt{freqs}. + \textbf{Hint:}
You may want to use the \texttt{numpy} \texttt{arange} function to
create \texttt{x\_coords}. Remember, \texttt{x\_coords} is the x-axis
and it should have points for each distinct language. + \textbf{Hint:}
To get \texttt{freqs}, you may want to use the \texttt{values()} method
on your result from step 3. That is,
\texttt{freqs\ =\ result\_from\_3.values()}. + \textbf{Hint:} To label
the x-axis you should use \texttt{plt.xticks(x\_coords,\ labels)} where
labels can be accessed through the \texttt{keys()} method on your result
from step 3.

    \begin{Verbatim}[commandchars=\\\{\}]
{\color{incolor}In [{\color{incolor}149}]:} \PY{k+kn}{from} \PY{n+nn}{collections} \PY{k}{import} \PY{n}{Counter}
          \PY{k+kn}{import} \PY{n+nn}{numpy} \PY{k}{as} \PY{n+nn}{np}
          \PY{k+kn}{import} \PY{n+nn}{matplotlib}\PY{n+nn}{.}\PY{n+nn}{pyplot} \PY{k}{as} \PY{n+nn}{plt}
          
          \PY{k}{with} \PY{n+nb}{open}\PY{p}{(}\PY{l+s+s1}{\PYZsq{}}\PY{l+s+s1}{languages.txt}\PY{l+s+s1}{\PYZsq{}}\PY{p}{,} \PY{l+s+s1}{\PYZsq{}}\PY{l+s+s1}{r}\PY{l+s+s1}{\PYZsq{}}\PY{p}{)} \PY{k}{as} \PY{n}{f}\PY{p}{:}
              \PY{c+c1}{\PYZsh{} Load the language strings from the file into a list}
              \PY{n}{lines} \PY{o}{=} \PY{n}{f}\PY{o}{.}\PY{n}{read}\PY{p}{(}\PY{p}{)}\PY{o}{.}\PY{n}{splitlines}\PY{p}{(}\PY{p}{)}
              
              \PY{c+c1}{\PYZsh{} Count the number occurrences of each element of the list}
              \PY{n}{freqs} \PY{o}{=} \PY{n}{Counter}\PY{p}{(}\PY{n}{lines}\PY{p}{)}
              \PY{n+nb}{print}\PY{p}{(}\PY{n}{freqs}\PY{p}{)}
              
              \PY{c+c1}{\PYZsh{} Createa bar plot to display the frequency of each language}
              \PY{n}{labels}\PY{p}{,} \PY{n}{values} \PY{o}{=} \PY{n+nb}{zip}\PY{p}{(}\PY{o}{*}\PY{n}{Counter}\PY{p}{(}\PY{n}{lines}\PY{p}{)}\PY{o}{.}\PY{n}{items}\PY{p}{(}\PY{p}{)}\PY{p}{)}
              \PY{n}{x\PYZus{}coords} \PY{o}{=} \PY{n}{np}\PY{o}{.}\PY{n}{arange}\PY{p}{(}\PY{n+nb}{len}\PY{p}{(}\PY{n}{labels}\PY{p}{)}\PY{p}{)}
          
              \PY{n}{plt}\PY{o}{.}\PY{n}{bar}\PY{p}{(}\PY{n}{x\PYZus{}coords}\PY{p}{,} \PY{n}{values}\PY{p}{,} \PY{l+m+mf}{0.5}\PY{p}{)}
              \PY{n}{plt}\PY{o}{.}\PY{n}{xticks}\PY{p}{(}\PY{n}{x\PYZus{}coords}\PY{p}{,} \PY{n}{labels}\PY{p}{)}
              \PY{n}{plt}\PY{o}{.}\PY{n}{show}\PY{p}{(}\PY{p}{)}
\end{Verbatim}


    \begin{Verbatim}[commandchars=\\\{\}]
Counter(\{'Python': 25, 'Java': 7, 'C++': 6, 'Other': 2, 'SQL': 2, 'C': 1, 'C\#': 1, 'VB': 1\})

    \end{Verbatim}

    \begin{center}
    \adjustimage{max size={0.9\linewidth}{0.9\paperheight}}{output_12_1.png}
    \end{center}
    { \hspace*{\fill} \\}
    
    \begin{center}\rule{0.5\linewidth}{\linethickness}\end{center}

 \#\#\# Problem 4 {[}10 pts{]}: Average Area of the Circles (Part 1) The
file \texttt{circles.txt} contains measurements of circle radii. Your
task is to write a script that reports the average area of the circles.
You will \textbf{not} use the \texttt{numpy} \texttt{mean} function.
Here are the requirements: 1. Open \texttt{circles.txt}, read in the
data, and convert the data to floats. Store in a list. 2. Write a
function that computes the area of a circle. The argument should be a
single \texttt{float}. 3. Write a function, called \texttt{myave}, that
computes the average of the areas of the circles. - At the very least,
\texttt{myave} should accept the list of radii as one argument and the
circle function that you wrote in step 2 as another argument. -
\textbf{Note:} There are other ways of doing this task, but I want you
to do it this way. 4. Print out the result.

    \begin{Verbatim}[commandchars=\\\{\}]
{\color{incolor}In [{\color{incolor}150}]:} \PY{n}{data} \PY{o}{=} \PY{p}{[}\PY{p}{]}
          \PY{k}{with} \PY{n+nb}{open}\PY{p}{(}\PY{l+s+s1}{\PYZsq{}}\PY{l+s+s1}{circles.txt}\PY{l+s+s1}{\PYZsq{}}\PY{p}{,} \PY{l+s+s1}{\PYZsq{}}\PY{l+s+s1}{r}\PY{l+s+s1}{\PYZsq{}}\PY{p}{)} \PY{k}{as} \PY{n}{f}\PY{p}{:}
              \PY{c+c1}{\PYZsh{} Convert the data to floats and store in a list}
              \PY{n}{data} \PY{o}{=} \PY{p}{[}\PY{n+nb}{float}\PY{p}{(}\PY{n}{item}\PY{p}{)} \PY{k}{for} \PY{n}{item} \PY{o+ow}{in} \PY{n}{f}\PY{o}{.}\PY{n}{read}\PY{p}{(}\PY{p}{)}\PY{o}{.}\PY{n}{splitlines}\PY{p}{(}\PY{p}{)}\PY{p}{]}
\end{Verbatim}


    \begin{Verbatim}[commandchars=\\\{\}]
{\color{incolor}In [{\color{incolor}151}]:} \PY{o}{\PYZpc{}}\PY{k}{matplotlib} inline
          \PY{k+kn}{from} \PY{n+nn}{collections} \PY{k}{import} \PY{n}{Counter}
          \PY{k+kn}{import} \PY{n+nn}{math}
          
          \PY{c+c1}{\PYZsh{} Write a function that ocmputes the area of a circle}
          \PY{k}{def} \PY{n+nf}{circle\PYZus{}area}\PY{p}{(}\PY{n}{radius}\PY{p}{)}\PY{p}{:}
              \PY{k}{return} \PY{n}{math}\PY{o}{.}\PY{n}{pi} \PY{o}{*} \PY{n}{radius} \PY{o}{*} \PY{n}{radius}
          
          \PY{c+c1}{\PYZsh{} Write a function that computes the average of the areas of the circle}
          \PY{k}{def} \PY{n+nf}{myave}\PY{p}{(}\PY{n}{radii}\PY{p}{,} \PY{n}{f}\PY{p}{)}\PY{p}{:}
              \PY{n}{areas} \PY{o}{=} \PY{p}{[}\PY{n}{f}\PY{p}{(}\PY{n}{radius}\PY{p}{)} \PY{k}{for} \PY{n}{radius} \PY{o+ow}{in} \PY{n}{radii}\PY{p}{]}
              \PY{k}{return} \PY{n+nb}{sum}\PY{p}{(}\PY{n}{areas}\PY{p}{)} \PY{o}{/} \PY{n+nb}{float}\PY{p}{(}\PY{n+nb}{len}\PY{p}{(}\PY{n}{areas}\PY{p}{)}\PY{p}{)}
          
          \PY{n+nb}{print}\PY{p}{(}\PY{n}{myave}\PY{p}{(}\PY{n}{data}\PY{p}{,} \PY{n}{circle\PYZus{}area}\PY{p}{)}\PY{p}{)}
\end{Verbatim}


    \begin{Verbatim}[commandchars=\\\{\}]
3.1958990970819956

    \end{Verbatim}

    \begin{center}\rule{0.5\linewidth}{\linethickness}\end{center}

 \#\# Part 3 {[}30 pts{]}: Closures and Decorators \#\#\# Problem 5
{[}10 pts{]}: Simple Bank Account Withdraw System The goal of this
problem is to write a simple bank account withdraw system. The problem
is based off of a problem in \emph{Structure and Interpretation of
Computer Programs}.

\textbf{Instructions:} Do each part in a different cell block and
clearly label each part.

    \textbf{a).} Write a closure to make withdraws from a bank account. The
outer function should be accept the initial balance as an argument (I'll
refer to this argument as \texttt{balance} in this problem statement,
but you can call it whatever you want). The inner function should accept
the withdraw amount as an argument and return the new balance.

\textbf{NOTE 1:} For this part, do not try to reassign \texttt{balance}
in the inner function. Just see what happens when you return a new
balance. You can store the new balance in a new name (call it
\texttt{new\_bal} if you want) or just return the new balance directly.

\textbf{NOTE 2:} You should check for basic exceptions (e.g. attempting
to withdraw more than the current balance).

Once you write your functions, demo them in your notebook as follows:

\begin{Shaded}
\begin{Highlighting}[]
\NormalTok{wd }\OperatorTok{=}\NormalTok{ make_withdraw(init_balance)}
\NormalTok{wd(withdraw_amount)}
\NormalTok{wd(new_withdraw_amount)}
\end{Highlighting}
\end{Shaded}

You should observe that this does not behave correctly. \textbf{Explain
why not.}

    \begin{Verbatim}[commandchars=\\\{\}]
{\color{incolor}In [{\color{incolor}142}]:} \PY{c+c1}{\PYZsh{}\PYZsh{}\PYZsh{} part a}
          \PY{k}{def} \PY{n+nf}{make\PYZus{}withdraw}\PY{p}{(}\PY{n}{balance}\PY{p}{)}\PY{p}{:}
              \PY{k}{def} \PY{n+nf}{update\PYZus{}balance}\PY{p}{(}\PY{n}{withdraw\PYZus{}amount}\PY{p}{)}\PY{p}{:}
                  \PY{n}{new\PYZus{}balance} \PY{o}{=} \PY{n}{balance} \PY{o}{\PYZhy{}} \PY{n}{withdraw\PYZus{}amount}
                  \PY{k}{if} \PY{n}{new\PYZus{}balance} \PY{o}{\PYZlt{}} \PY{l+m+mi}{0}\PY{p}{:}
                      \PY{n+nb}{print}\PY{p}{(}\PY{l+s+s2}{\PYZdq{}}\PY{l+s+s2}{You have not enough balance}\PY{l+s+s2}{\PYZdq{}}\PY{p}{)}
                      \PY{k}{return}
                  \PY{k}{return} \PY{n}{new\PYZus{}balance}
              \PY{k}{return} \PY{n}{update\PYZus{}balance}
          
          \PY{c+c1}{\PYZsh{} Initialize variables}
          \PY{n}{init\PYZus{}balance} \PY{o}{=} \PY{l+m+mi}{1000}
          \PY{n}{withdraw\PYZus{}amount} \PY{o}{=} \PY{l+m+mi}{100}
          \PY{n}{new\PYZus{}withdraw\PYZus{}amount} \PY{o}{=} \PY{l+m+mi}{200}
          
          \PY{c+c1}{\PYZsh{} Demo functions}
          \PY{n}{wd} \PY{o}{=} \PY{n}{make\PYZus{}withdraw}\PY{p}{(}\PY{n}{init\PYZus{}balance}\PY{p}{)}
          \PY{n}{wd}\PY{p}{(}\PY{n}{withdraw\PYZus{}amount}\PY{p}{)}
          \PY{n}{wd}\PY{p}{(}\PY{n}{new\PYZus{}withdraw\PYZus{}amount}\PY{p}{)}
\end{Verbatim}


\begin{Verbatim}[commandchars=\\\{\}]
{\color{outcolor}Out[{\color{outcolor}142}]:} 800
\end{Verbatim}
            
    We want the new balance after withdrawing 200 to be 1000 - 100 - 200 =
700, but as we can see, the new balance is 800. This is because
everytime update\_balance is called, balance is equal to 1000, the
init\_balance.

    \textbf{b).} You can fix things up by updating \texttt{balance} within
the inner function. But this won't work. Try it out and explain why it
doesn't work. Try to use the language that we used in lecture.

\begin{itemize}
\tightlist
\item
  \textbf{Hint:}
  \href{https://docs.python.org/3/reference/executionmodel.html}{Python
  Execution Model}.
\end{itemize}

    \begin{Verbatim}[commandchars=\\\{\}]
{\color{incolor}In [{\color{incolor}143}]:} \PY{c+c1}{\PYZsh{}\PYZsh{}\PYZsh{} part b}
          \PY{k}{def} \PY{n+nf}{make\PYZus{}withdraw}\PY{p}{(}\PY{n}{balance}\PY{p}{)}\PY{p}{:}
              \PY{k}{def} \PY{n+nf}{update\PYZus{}balance}\PY{p}{(}\PY{n}{withdraw\PYZus{}amount}\PY{p}{)}\PY{p}{:}
                  \PY{n}{balance} \PY{o}{=} \PY{n}{balance} \PY{o}{\PYZhy{}} \PY{n}{withdraw\PYZus{}amount}
                  \PY{k}{if} \PY{n}{balance} \PY{o}{\PYZlt{}} \PY{l+m+mi}{0}\PY{p}{:}
                      \PY{n+nb}{print}\PY{p}{(}\PY{l+s+s2}{\PYZdq{}}\PY{l+s+s2}{You have not enough balance}\PY{l+s+s2}{\PYZdq{}}\PY{p}{)}
                      \PY{k}{return}
                  \PY{k}{return} \PY{n}{balance}
              \PY{k}{return} \PY{n}{update\PYZus{}balance}
          
          \PY{c+c1}{\PYZsh{} Initialize variables}
          \PY{n}{init\PYZus{}balance} \PY{o}{=} \PY{l+m+mi}{1000}
          \PY{n}{withdraw\PYZus{}amount} \PY{o}{=} \PY{l+m+mi}{100}
          \PY{n}{new\PYZus{}withdraw\PYZus{}amount} \PY{o}{=} \PY{l+m+mi}{200}
          
          \PY{c+c1}{\PYZsh{} Demo functions}
          \PY{n}{wd} \PY{o}{=} \PY{n}{make\PYZus{}withdraw}\PY{p}{(}\PY{n}{init\PYZus{}balance}\PY{p}{)}
          \PY{n}{wd}\PY{p}{(}\PY{n}{withdraw\PYZus{}amount}\PY{p}{)}
          \PY{n}{wd}\PY{p}{(}\PY{n}{new\PYZus{}withdraw\PYZus{}amount}\PY{p}{)}
\end{Verbatim}


    \begin{Verbatim}[commandchars=\\\{\}]

        ---------------------------------------------------------------------------

        UnboundLocalError                         Traceback (most recent call last)

        <ipython-input-143-8ecc8cc27173> in <module>()
         16 \# Demo functions
         17 wd = make\_withdraw(init\_balance)
    ---> 18 wd(withdraw\_amount)
         19 wd(new\_withdraw\_amount)


        <ipython-input-143-8ecc8cc27173> in update\_balance(withdraw\_amount)
          2 def make\_withdraw(balance):
          3     def update\_balance(withdraw\_amount):
    ----> 4         balance = balance - withdraw\_amount
          5         if balance < 0:
          6             print("You have not enough balance")


        UnboundLocalError: local variable 'balance' referenced before assignment

    \end{Verbatim}

    As we can see, \texttt{balance} cannot be updated within the inner
function. As mentioned in the lecture, it is because as soon as
\texttt{balance} is captured by the inner function, it cannot be
changed: "we have lost direct access to its manipulation." This is
called encapsulation.

    \textbf{c).} Now make just one small change to your code from Part b.
Declare \texttt{balance} as a nonlocal variable using the
\texttt{nonlocal} keyword. That is, before you get to the inner
function, say \texttt{nonlocal\ balance}. Here's some information on the
\texttt{nonlocal} statement:
\href{https://docs.python.org/3/reference/simple_stmts.html\#nonlocal}{\texttt{nonlocal}}.

Now test things out like you did in Part a. It should be behaving
correctly now.

    \begin{Verbatim}[commandchars=\\\{\}]
{\color{incolor}In [{\color{incolor}144}]:} \PY{c+c1}{\PYZsh{}\PYZsh{}\PYZsh{} part c}
          \PY{k}{def} \PY{n+nf}{make\PYZus{}withdraw}\PY{p}{(}\PY{n}{balance}\PY{p}{)}\PY{p}{:}
              \PY{k}{def} \PY{n+nf}{update\PYZus{}balance}\PY{p}{(}\PY{n}{withdraw\PYZus{}amount}\PY{p}{)}\PY{p}{:}
                  \PY{c+c1}{\PYZsh{} Declare balance as a nonlocal variable}
                  \PY{k}{nonlocal} \PY{n}{balance}
                  \PY{k}{if} \PY{n}{balance} \PY{o}{\PYZlt{}} \PY{n}{withdraw\PYZus{}amount}\PY{p}{:}
                      \PY{n+nb}{print}\PY{p}{(}\PY{l+s+s2}{\PYZdq{}}\PY{l+s+s2}{You have not enough balance}\PY{l+s+s2}{\PYZdq{}}\PY{p}{)}
                      \PY{k}{return}
                  \PY{n}{balance} \PY{o}{=} \PY{n}{balance} \PY{o}{\PYZhy{}} \PY{n}{withdraw\PYZus{}amount}
                  \PY{k}{return} \PY{n}{balance}
              \PY{k}{return} \PY{n}{update\PYZus{}balance}
          
          \PY{c+c1}{\PYZsh{} Initialize variables}
          \PY{n}{init\PYZus{}balance} \PY{o}{=} \PY{l+m+mi}{1000}
          \PY{n}{withdraw\PYZus{}amount} \PY{o}{=} \PY{l+m+mi}{100}
          \PY{n}{new\PYZus{}withdraw\PYZus{}amount} \PY{o}{=} \PY{l+m+mi}{200}
          
          \PY{c+c1}{\PYZsh{} Demo functions}
          \PY{n}{wd} \PY{o}{=} \PY{n}{make\PYZus{}withdraw}\PY{p}{(}\PY{n}{init\PYZus{}balance}\PY{p}{)}
          \PY{n}{wd}\PY{p}{(}\PY{n}{withdraw\PYZus{}amount}\PY{p}{)}
          \PY{n}{wd}\PY{p}{(}\PY{n}{new\PYZus{}withdraw\PYZus{}amount}\PY{p}{)}
\end{Verbatim}


\begin{Verbatim}[commandchars=\\\{\}]
{\color{outcolor}Out[{\color{outcolor}144}]:} 700
\end{Verbatim}
            
    \textbf{d).} Finally, visualize your code with
\href{http://pythontutor.com/}{Python Tutor} and embed your
visualization in your notebook. Pay attention to the variable
\texttt{balance}.

    \begin{Verbatim}[commandchars=\\\{\}]
{\color{incolor}In [{\color{incolor}145}]:} \PY{k+kn}{from} \PY{n+nn}{IPython}\PY{n+nn}{.}\PY{n+nn}{core}\PY{n+nn}{.}\PY{n+nn}{display} \PY{k}{import} \PY{n}{display}\PY{p}{,} \PY{n}{HTML}
          
          \PY{n}{display}\PY{p}{(}\PY{n}{HTML}\PY{p}{(}\PY{l+s+s1}{\PYZsq{}}\PY{l+s+s1}{\PYZlt{}iframe width=}\PY{l+s+s1}{\PYZdq{}}\PY{l+s+s1}{800}\PY{l+s+s1}{\PYZdq{}}\PY{l+s+s1}{ height=}\PY{l+s+s1}{\PYZdq{}}\PY{l+s+s1}{500}\PY{l+s+s1}{\PYZdq{}}\PY{l+s+s1}{ frameborder=}\PY{l+s+s1}{\PYZdq{}}\PY{l+s+s1}{0}\PY{l+s+s1}{\PYZdq{}}\PY{l+s+s1}{ src=}\PY{l+s+s1}{\PYZdq{}}\PY{l+s+s1}{http://pythontutor.com/iframe\PYZhy{}embed.html\PYZsh{}code=}\PY{l+s+si}{\PYZpc{}23\PYZpc{}}\PY{l+s+s1}{23}\PY{l+s+si}{\PYZpc{}23\PYZpc{}}\PY{l+s+s1}{20part}\PY{l+s+si}{\PYZpc{}20c}\PY{l+s+s1}{\PYZpc{}}\PY{l+s+s1}{0Adef}\PY{l+s+s1}{\PYZpc{}}\PY{l+s+s1}{20make\PYZus{}withdraw}\PY{l+s+s1}{\PYZpc{}}\PY{l+s+s1}{28balance}\PY{l+s+si}{\PYZpc{}29\PYZpc{}}\PY{l+s+s1}{3A}\PY{l+s+s1}{\PYZpc{}}\PY{l+s+s1}{0A}\PY{l+s+si}{\PYZpc{}20\PYZpc{}}\PY{l+s+s1}{20}\PY{l+s+si}{\PYZpc{}20\PYZpc{}}\PY{l+s+s1}{20def}\PY{l+s+si}{\PYZpc{}20u}\PY{l+s+s1}{pdate\PYZus{}balance}\PY{l+s+s1}{\PYZpc{}}\PY{l+s+s1}{28withdraw\PYZus{}amount}\PY{l+s+si}{\PYZpc{}29\PYZpc{}}\PY{l+s+s1}{3A}\PY{l+s+s1}{\PYZpc{}}\PY{l+s+s1}{0A}\PY{l+s+si}{\PYZpc{}20\PYZpc{}}\PY{l+s+s1}{20}\PY{l+s+si}{\PYZpc{}20\PYZpc{}}\PY{l+s+s1}{20}\PY{l+s+si}{\PYZpc{}20\PYZpc{}}\PY{l+s+s1}{20}\PY{l+s+si}{\PYZpc{}20\PYZpc{}}\PY{l+s+s1}{20}\PY{l+s+si}{\PYZpc{}23\PYZpc{}}\PY{l+s+s1}{20Declare}\PY{l+s+s1}{\PYZpc{}}\PY{l+s+s1}{20balance}\PY{l+s+s1}{\PYZpc{}}\PY{l+s+s1}{20as}\PY{l+s+s1}{\PYZpc{}}\PY{l+s+s1}{20a}\PY{l+s+s1}{\PYZpc{}}\PY{l+s+s1}{20nonlocal}\PY{l+s+s1}{\PYZpc{}}\PY{l+s+s1}{20variable}\PY{l+s+s1}{\PYZpc{}}\PY{l+s+s1}{0A}\PY{l+s+si}{\PYZpc{}20\PYZpc{}}\PY{l+s+s1}{20}\PY{l+s+si}{\PYZpc{}20\PYZpc{}}\PY{l+s+s1}{20}\PY{l+s+si}{\PYZpc{}20\PYZpc{}}\PY{l+s+s1}{20}\PY{l+s+si}{\PYZpc{}20\PYZpc{}}\PY{l+s+s1}{20nonlocal}\PY{l+s+s1}{\PYZpc{}}\PY{l+s+s1}{20balance}\PY{l+s+s1}{\PYZpc{}}\PY{l+s+s1}{0A}\PY{l+s+si}{\PYZpc{}20\PYZpc{}}\PY{l+s+s1}{20}\PY{l+s+si}{\PYZpc{}20\PYZpc{}}\PY{l+s+s1}{20}\PY{l+s+si}{\PYZpc{}20\PYZpc{}}\PY{l+s+s1}{20}\PY{l+s+si}{\PYZpc{}20\PYZpc{}}\PY{l+s+s1}{20if}\PY{l+s+s1}{\PYZpc{}}\PY{l+s+s1}{20balance}\PY{l+s+si}{\PYZpc{}20\PYZpc{}}\PY{l+s+s1}{3C}\PY{l+s+s1}{\PYZpc{}}\PY{l+s+s1}{20withdraw\PYZus{}amount}\PY{l+s+s1}{\PYZpc{}}\PY{l+s+s1}{3A}\PY{l+s+s1}{\PYZpc{}}\PY{l+s+s1}{0A}\PY{l+s+si}{\PYZpc{}20\PYZpc{}}\PY{l+s+s1}{20}\PY{l+s+si}{\PYZpc{}20\PYZpc{}}\PY{l+s+s1}{20}\PY{l+s+si}{\PYZpc{}20\PYZpc{}}\PY{l+s+s1}{20}\PY{l+s+si}{\PYZpc{}20\PYZpc{}}\PY{l+s+s1}{20}\PY{l+s+si}{\PYZpc{}20\PYZpc{}}\PY{l+s+s1}{20}\PY{l+s+si}{\PYZpc{}20\PYZpc{}}\PY{l+s+s1}{20print}\PY{l+s+si}{\PYZpc{}28\PYZpc{}}\PY{l+s+s1}{22You}\PY{l+s+s1}{\PYZpc{}}\PY{l+s+s1}{20have}\PY{l+s+s1}{\PYZpc{}}\PY{l+s+s1}{20not}\PY{l+s+si}{\PYZpc{}20e}\PY{l+s+s1}{nough}\PY{l+s+s1}{\PYZpc{}}\PY{l+s+s1}{20balance}\PY{l+s+si}{\PYZpc{}22\PYZpc{}}\PY{l+s+s1}{29}\PY{l+s+s1}{\PYZpc{}}\PY{l+s+s1}{0A}\PY{l+s+si}{\PYZpc{}20\PYZpc{}}\PY{l+s+s1}{20}\PY{l+s+si}{\PYZpc{}20\PYZpc{}}\PY{l+s+s1}{20}\PY{l+s+si}{\PYZpc{}20\PYZpc{}}\PY{l+s+s1}{20}\PY{l+s+si}{\PYZpc{}20\PYZpc{}}\PY{l+s+s1}{20}\PY{l+s+si}{\PYZpc{}20\PYZpc{}}\PY{l+s+s1}{20}\PY{l+s+si}{\PYZpc{}20\PYZpc{}}\PY{l+s+s1}{20return}\PY{l+s+s1}{\PYZpc{}}\PY{l+s+s1}{0A}\PY{l+s+si}{\PYZpc{}20\PYZpc{}}\PY{l+s+s1}{20}\PY{l+s+si}{\PYZpc{}20\PYZpc{}}\PY{l+s+s1}{20}\PY{l+s+si}{\PYZpc{}20\PYZpc{}}\PY{l+s+s1}{20}\PY{l+s+si}{\PYZpc{}20\PYZpc{}}\PY{l+s+s1}{20balance}\PY{l+s+si}{\PYZpc{}20\PYZpc{}}\PY{l+s+s1}{3D}\PY{l+s+s1}{\PYZpc{}}\PY{l+s+s1}{20balance}\PY{l+s+s1}{\PYZpc{}}\PY{l+s+s1}{20\PYZhy{}}\PY{l+s+s1}{\PYZpc{}}\PY{l+s+s1}{20withdraw\PYZus{}amount}\PY{l+s+s1}{\PYZpc{}}\PY{l+s+s1}{0A}\PY{l+s+si}{\PYZpc{}20\PYZpc{}}\PY{l+s+s1}{20}\PY{l+s+si}{\PYZpc{}20\PYZpc{}}\PY{l+s+s1}{20}\PY{l+s+si}{\PYZpc{}20\PYZpc{}}\PY{l+s+s1}{20}\PY{l+s+si}{\PYZpc{}20\PYZpc{}}\PY{l+s+s1}{20return}\PY{l+s+s1}{\PYZpc{}}\PY{l+s+s1}{20balance}\PY{l+s+s1}{\PYZpc{}}\PY{l+s+s1}{0A}\PY{l+s+si}{\PYZpc{}20\PYZpc{}}\PY{l+s+s1}{20}\PY{l+s+si}{\PYZpc{}20\PYZpc{}}\PY{l+s+s1}{20return}\PY{l+s+si}{\PYZpc{}20u}\PY{l+s+s1}{pdate\PYZus{}balance}\PY{l+s+s1}{\PYZpc{}}\PY{l+s+s1}{0A}\PY{l+s+s1}{\PYZpc{}}\PY{l+s+s1}{0A}\PY{l+s+si}{\PYZpc{}23\PYZpc{}}\PY{l+s+s1}{20Initialize}\PY{l+s+s1}{\PYZpc{}}\PY{l+s+s1}{20variables}\PY{l+s+s1}{\PYZpc{}}\PY{l+s+s1}{0Ainit\PYZus{}balance}\PY{l+s+si}{\PYZpc{}20\PYZpc{}}\PY{l+s+s1}{3D}\PY{l+s+si}{\PYZpc{}201000\PYZpc{}}\PY{l+s+s1}{0Awithdraw\PYZus{}amount}\PY{l+s+si}{\PYZpc{}20\PYZpc{}}\PY{l+s+s1}{3D}\PY{l+s+si}{\PYZpc{}20100\PYZpc{}}\PY{l+s+s1}{0Anew\PYZus{}withdraw\PYZus{}amount}\PY{l+s+si}{\PYZpc{}20\PYZpc{}}\PY{l+s+s1}{3D}\PY{l+s+si}{\PYZpc{}20200\PYZpc{}}\PY{l+s+s1}{0A}\PY{l+s+s1}{\PYZpc{}}\PY{l+s+s1}{0A}\PY{l+s+si}{\PYZpc{}23\PYZpc{}}\PY{l+s+s1}{20Demo}\PY{l+s+si}{\PYZpc{}20f}\PY{l+s+s1}{unctions}\PY{l+s+s1}{\PYZpc{}}\PY{l+s+s1}{0Awd}\PY{l+s+si}{\PYZpc{}20\PYZpc{}}\PY{l+s+s1}{3D}\PY{l+s+s1}{\PYZpc{}}\PY{l+s+s1}{20make\PYZus{}withdraw}\PY{l+s+si}{\PYZpc{}28i}\PY{l+s+s1}{nit\PYZus{}balance}\PY{l+s+si}{\PYZpc{}29\PYZpc{}}\PY{l+s+s1}{0Awd}\PY{l+s+s1}{\PYZpc{}}\PY{l+s+s1}{28withdraw\PYZus{}amount}\PY{l+s+si}{\PYZpc{}29\PYZpc{}}\PY{l+s+s1}{0Awd}\PY{l+s+s1}{\PYZpc{}}\PY{l+s+s1}{28new\PYZus{}withdraw\PYZus{}amount}\PY{l+s+si}{\PYZpc{}29\PYZpc{}}\PY{l+s+s1}{0A\PYZam{}codeDivHeight=400\PYZam{}codeDivWidth=350\PYZam{}cumulative=false\PYZam{}curInstr=0\PYZam{}heapPrimitives=nevernest\PYZam{}origin=opt\PYZhy{}frontend.js\PYZam{}py=3\PYZam{}rawInputLstJSON=}\PY{l+s+s1}{\PYZpc{}}\PY{l+s+s1}{5B}\PY{l+s+s1}{\PYZpc{}}\PY{l+s+s1}{5D\PYZam{}textReferences=false}\PY{l+s+s1}{\PYZdq{}}\PY{l+s+s1}{\PYZgt{} \PYZlt{}/iframe\PYZgt{}}\PY{l+s+s1}{\PYZsq{}}\PY{p}{)}\PY{p}{)}
\end{Verbatim}


    
    \begin{verbatim}
<IPython.core.display.HTML object>
    \end{verbatim}

    
    \begin{center}\rule{0.5\linewidth}{\linethickness}\end{center}

 \#\#\# Problem 6 {[}10 pts{]}: Average Area of the Circles (part 2)
Let's return to the data from Problem 4.

Write two functions: 1. The first function should return the average
circle radius (you can re-use the one you already wrote if you'd like,
but you might need to update it slightly for this problem). 2. The
second function should just use \texttt{numpy} to compute the average.

Write a decorator to time the evaluation of each function. You can use
the timing decorator from lecture.

\paragraph{Notes and Hints}\label{notes-and-hints}

\begin{enumerate}
\def\labelenumi{\arabic{enumi}.}
\tightlist
\item
  Be fair!
\item
  To be as fair as possible, do the following:
\item
  Create an \texttt{areas} list/array \emph{outside} of your averaging
  functions. This means that you should do a loop over the radii you
  read in from \texttt{circles.txt}, compute the area from each point,
  and store that area in an array. Do you know why this is more fair? If
  not, think about it make sure you understand. Also, do not use
  \texttt{append}. Instead, preallocate space for your \texttt{area}
  list/array.
\item
  Your \texttt{my\_ave} function should accept your areas data as a
  list. Remember, to allocate a list you should do \texttt{{[}0.0{]}*N}:
  if you use such a construct your list will be filled in with \(N\)
  zeros.
\item
  Your \texttt{np\_ave} function should accept your areas data as a
  \texttt{numpy} array. To allocate a \texttt{numpy} array do
  \texttt{areas\_np\ =\ np.zeros(len(radii))}.
\item
  Now both functions are using the best data types possible for their
  tasks.
\end{enumerate}

    \begin{Verbatim}[commandchars=\\\{\}]
{\color{incolor}In [{\color{incolor}155}]:} \PY{k+kn}{from} \PY{n+nn}{collections} \PY{k}{import} \PY{n}{Counter}
          \PY{k+kn}{import} \PY{n+nn}{math}
          
          \PY{n}{radii} \PY{o}{=} \PY{p}{[}\PY{p}{]}
          \PY{k}{with} \PY{n+nb}{open}\PY{p}{(}\PY{l+s+s1}{\PYZsq{}}\PY{l+s+s1}{circles.txt}\PY{l+s+s1}{\PYZsq{}}\PY{p}{,} \PY{l+s+s1}{\PYZsq{}}\PY{l+s+s1}{r}\PY{l+s+s1}{\PYZsq{}}\PY{p}{)} \PY{k}{as} \PY{n}{f}\PY{p}{:}
              \PY{c+c1}{\PYZsh{} Convert the data to floats and store in a list}
              \PY{n}{radii} \PY{o}{=} \PY{p}{[}\PY{n+nb}{float}\PY{p}{(}\PY{n}{item}\PY{p}{)} \PY{k}{for} \PY{n}{item} \PY{o+ow}{in} \PY{n}{f}\PY{o}{.}\PY{n}{read}\PY{p}{(}\PY{p}{)}\PY{o}{.}\PY{n}{splitlines}\PY{p}{(}\PY{p}{)}\PY{p}{]}
\end{Verbatim}


    \begin{Verbatim}[commandchars=\\\{\}]
{\color{incolor}In [{\color{incolor}156}]:} \PY{c+c1}{\PYZsh{} Timer function from lecture 4}
          \PY{k+kn}{import} \PY{n+nn}{time}
          \PY{k}{def} \PY{n+nf}{timer}\PY{p}{(}\PY{n}{f}\PY{p}{)}\PY{p}{:}
              \PY{k}{def} \PY{n+nf}{inner}\PY{p}{(}\PY{o}{*}\PY{n}{args}\PY{p}{)}\PY{p}{:}
                  \PY{n}{t0} \PY{o}{=} \PY{n}{time}\PY{o}{.}\PY{n}{time}\PY{p}{(}\PY{p}{)}
                  \PY{n}{output} \PY{o}{=} \PY{n}{f}\PY{p}{(}\PY{o}{*}\PY{n}{args}\PY{p}{)}
                  \PY{n}{elapsed} \PY{o}{=} \PY{n}{time}\PY{o}{.}\PY{n}{time}\PY{p}{(}\PY{p}{)} \PY{o}{\PYZhy{}} \PY{n}{t0}
                  \PY{n+nb}{print}\PY{p}{(}\PY{l+s+s2}{\PYZdq{}}\PY{l+s+s2}{Time Elapsed}\PY{l+s+s2}{\PYZdq{}}\PY{p}{,} \PY{n}{elapsed}\PY{p}{)}
                  \PY{k}{return} \PY{n}{output}
              \PY{k}{return} \PY{n}{inner}
\end{Verbatim}


    \begin{Verbatim}[commandchars=\\\{\}]
{\color{incolor}In [{\color{incolor}167}]:} \PY{c+c1}{\PYZsh{} function that computes the area of a circle}
          \PY{k}{def} \PY{n+nf}{circle\PYZus{}area}\PY{p}{(}\PY{n}{radius}\PY{p}{)}\PY{p}{:}
              \PY{k}{return} \PY{n}{math}\PY{o}{.}\PY{n}{pi} \PY{o}{*} \PY{n}{radius} \PY{o}{*} \PY{n}{radius}
          
          \PY{n}{areas} \PY{o}{=} \PY{p}{[}\PY{l+m+mf}{0.0}\PY{p}{]} \PY{o}{*} \PY{n+nb}{len}\PY{p}{(}\PY{n}{radii}\PY{p}{)}
          \PY{n}{areas} \PY{o}{=} \PY{p}{[}\PY{n}{circle\PYZus{}area}\PY{p}{(}\PY{n}{radius}\PY{p}{)} \PY{k}{for} \PY{n}{radius} \PY{o+ow}{in} \PY{n}{radii}\PY{p}{]}
          
          \PY{n}{areas\PYZus{}np} \PY{o}{=} \PY{n}{np}\PY{o}{.}\PY{n}{zeros}\PY{p}{(}\PY{n+nb}{len}\PY{p}{(}\PY{n}{radii}\PY{p}{)}\PY{p}{)}
          \PY{n}{areas\PYZus{}np} \PY{o}{=} \PY{n}{np}\PY{o}{.}\PY{n}{array}\PY{p}{(}\PY{n}{areas}\PY{p}{)}
\end{Verbatim}


    \begin{Verbatim}[commandchars=\\\{\}]
{\color{incolor}In [{\color{incolor}186}]:} \PY{n+nd}{@timer}
          \PY{k}{def} \PY{n+nf}{myave}\PY{p}{(}\PY{n}{areas}\PY{p}{)}\PY{p}{:}
              \PY{k}{return} \PY{n+nb}{sum}\PY{p}{(}\PY{n}{areas}\PY{p}{)} \PY{o}{/} \PY{n+nb}{float}\PY{p}{(}\PY{n+nb}{len}\PY{p}{(}\PY{n}{areas}\PY{p}{)}\PY{p}{)}
          
          \PY{n+nd}{@timer}
          \PY{k}{def} \PY{n+nf}{npave}\PY{p}{(}\PY{n}{areas}\PY{p}{)}\PY{p}{:}
              \PY{k}{return} \PY{n}{np}\PY{o}{.}\PY{n}{mean}\PY{p}{(}\PY{n}{areas}\PY{p}{)}
          
          \PY{n}{myave}\PY{p}{(}\PY{n}{areas}\PY{p}{)}
          \PY{n}{npave}\PY{p}{(}\PY{n}{areas\PYZus{}np}\PY{p}{)}
\end{Verbatim}


    \begin{Verbatim}[commandchars=\\\{\}]
Time Elapsed 8.106231689453125e-06
Time Elapsed 6.723403930664062e-05

    \end{Verbatim}

\begin{Verbatim}[commandchars=\\\{\}]
{\color{outcolor}Out[{\color{outcolor}186}]:} 3.195899097081994
\end{Verbatim}
            
    \begin{center}\rule{0.5\linewidth}{\linethickness}\end{center}

 \#\#\# Problem 7 {[}10 pts{]}: Positivity Write a decorator to check if
a quantity returned from a function is positive. An exception should be
raised if the quantity is not positive.

Write three functions and decorate them with your decorator: 1. A
function that returns the discriminant \(\displaystyle d = b^{2} - 4ac\)
2. A function that computes the absolute value (you must write this
yourself...do not use built-ins) 3. A function of your own choosing.

Show that your decorator behaves correctly. That is, for each function,
show two cases (where applicable): 1. A case where positivity is not
violated 2. A case where positivity is violated

    \begin{Verbatim}[commandchars=\\\{\}]
{\color{incolor}In [{\color{incolor}231}]:} \PY{c+c1}{\PYZsh{} Decorator to check if a quntity returned from a function is positive}
          \PY{k}{def} \PY{n+nf}{check\PYZus{}positivity}\PY{p}{(}\PY{n}{f}\PY{p}{)}\PY{p}{:}
              \PY{k}{def} \PY{n+nf}{inner}\PY{p}{(}\PY{o}{*}\PY{n}{args}\PY{p}{)}\PY{p}{:}
                  \PY{n}{output} \PY{o}{=} \PY{n}{f}\PY{p}{(}\PY{o}{*}\PY{n}{args}\PY{p}{)}
                  \PY{k}{if} \PY{n}{output} \PY{o}{\PYZlt{}} \PY{l+m+mi}{0}\PY{p}{:}
                      \PY{k}{raise} \PY{n+ne}{ValueError}\PY{p}{(}\PY{l+s+s1}{\PYZsq{}}\PY{l+s+s1}{Output is not positive}\PY{l+s+s1}{\PYZsq{}}\PY{p}{)}
                  \PY{k}{return} \PY{n}{output}
              \PY{k}{return} \PY{n}{inner}
\end{Verbatim}


    \begin{Verbatim}[commandchars=\\\{\}]
{\color{incolor}In [{\color{incolor}232}]:} \PY{c+c1}{\PYZsh{} A function that returns the discriminant}
          \PY{n+nd}{@check\PYZus{}positivity}
          \PY{k}{def} \PY{n+nf}{calc\PYZus{}discriminant}\PY{p}{(}\PY{n}{a}\PY{p}{,}\PY{n}{b}\PY{p}{,}\PY{n}{c}\PY{p}{)}\PY{p}{:}
              \PY{k}{return} \PY{n}{b} \PY{o}{*} \PY{n}{b} \PY{o}{\PYZhy{}} \PY{l+m+mi}{4} \PY{o}{*} \PY{n}{a} \PY{o}{*} \PY{n}{c}
          
          \PY{c+c1}{\PYZsh{} A function that computes the absolute value}
          \PY{n+nd}{@check\PYZus{}positivity}
          \PY{k}{def} \PY{n+nf}{calc\PYZus{}absolute}\PY{p}{(}\PY{n}{a}\PY{p}{)}\PY{p}{:}
              \PY{k}{if} \PY{n}{a} \PY{o}{\PYZgt{}}\PY{o}{=} \PY{l+m+mi}{0}\PY{p}{:}
                  \PY{k}{return} \PY{n}{a}
              \PY{k}{else}\PY{p}{:}
                  \PY{k}{return} \PY{o}{\PYZhy{}}\PY{n}{a}
          
          \PY{c+c1}{\PYZsh{} A function that computes the sum of two values}
          \PY{n+nd}{@check\PYZus{}positivity}
          \PY{k}{def} \PY{n+nf}{calc\PYZus{}sum}\PY{p}{(}\PY{n}{a}\PY{p}{,}\PY{n}{b}\PY{p}{)}\PY{p}{:}
              \PY{k}{return} \PY{n}{a} \PY{o}{+} \PY{n}{b}
\end{Verbatim}


    \begin{Verbatim}[commandchars=\\\{\}]
{\color{incolor}In [{\color{incolor}257}]:} \PY{k+kn}{import} \PY{n+nn}{unittest}
          
          \PY{k}{class} \PY{n+nc}{Test}\PY{p}{(}\PY{n}{unittest}\PY{o}{.}\PY{n}{TestCase}\PY{p}{)}\PY{p}{:}
          
              \PY{c+c1}{\PYZsh{} Test cases where functions will not raise exceptions}
              \PY{k}{def} \PY{n+nf}{test\PYZus{}success}\PY{p}{(}\PY{n+nb+bp}{self}\PY{p}{)}\PY{p}{:}
                  \PY{c+c1}{\PYZsh{} Test discriminant}
                  \PY{n+nb+bp}{self}\PY{o}{.}\PY{n}{assertEqual}\PY{p}{(}\PY{n}{calc\PYZus{}discriminant}\PY{p}{(}\PY{l+m+mi}{0}\PY{p}{,}\PY{l+m+mi}{0}\PY{p}{,}\PY{l+m+mi}{0}\PY{p}{)}\PY{p}{,} \PY{l+m+mi}{0}\PY{p}{)}
                  \PY{n+nb+bp}{self}\PY{o}{.}\PY{n}{assertEqual}\PY{p}{(}\PY{n}{calc\PYZus{}discriminant}\PY{p}{(}\PY{l+m+mi}{1}\PY{p}{,}\PY{l+m+mi}{1}\PY{p}{,}\PY{l+m+mi}{0}\PY{p}{)}\PY{p}{,} \PY{l+m+mi}{1}\PY{p}{)}
                  \PY{n+nb+bp}{self}\PY{o}{.}\PY{n}{assertEqual}\PY{p}{(}\PY{n}{calc\PYZus{}discriminant}\PY{p}{(}\PY{l+m+mi}{1}\PY{p}{,}\PY{l+m+mi}{2}\PY{p}{,}\PY{l+m+mi}{1}\PY{p}{)}\PY{p}{,} \PY{l+m+mi}{0}\PY{p}{)}
                  
                  \PY{c+c1}{\PYZsh{} Test absolute value}
                  \PY{n+nb+bp}{self}\PY{o}{.}\PY{n}{assertEqual}\PY{p}{(}\PY{n}{calc\PYZus{}absolute}\PY{p}{(}\PY{l+m+mi}{0}\PY{p}{)}\PY{p}{,} \PY{l+m+mi}{0}\PY{p}{)}
                  \PY{n+nb+bp}{self}\PY{o}{.}\PY{n}{assertEqual}\PY{p}{(}\PY{n}{calc\PYZus{}absolute}\PY{p}{(}\PY{l+m+mi}{1}\PY{p}{)}\PY{p}{,} \PY{l+m+mi}{1}\PY{p}{)}
                  \PY{n+nb+bp}{self}\PY{o}{.}\PY{n}{assertEqual}\PY{p}{(}\PY{n}{calc\PYZus{}absolute}\PY{p}{(}\PY{o}{\PYZhy{}}\PY{l+m+mi}{1}\PY{p}{)}\PY{p}{,} \PY{l+m+mi}{1}\PY{p}{)}
                  
                  \PY{c+c1}{\PYZsh{} Test sum}
                  \PY{n+nb+bp}{self}\PY{o}{.}\PY{n}{assertEqual}\PY{p}{(}\PY{n}{calc\PYZus{}sum}\PY{p}{(}\PY{l+m+mi}{0}\PY{p}{,} \PY{l+m+mi}{0}\PY{p}{)}\PY{p}{,} \PY{l+m+mi}{0}\PY{p}{)}
                  \PY{n+nb+bp}{self}\PY{o}{.}\PY{n}{assertEqual}\PY{p}{(}\PY{n}{calc\PYZus{}sum}\PY{p}{(}\PY{l+m+mi}{1}\PY{p}{,} \PY{l+m+mi}{0}\PY{p}{)}\PY{p}{,} \PY{l+m+mi}{1}\PY{p}{)}
                  \PY{n+nb+bp}{self}\PY{o}{.}\PY{n}{assertEqual}\PY{p}{(}\PY{n}{calc\PYZus{}sum}\PY{p}{(}\PY{l+m+mi}{2}\PY{p}{,} \PY{o}{\PYZhy{}}\PY{l+m+mi}{1}\PY{p}{)}\PY{p}{,} \PY{l+m+mi}{1}\PY{p}{)}
          
              \PY{c+c1}{\PYZsh{} Test cases where functions will raise exceptions}
              \PY{k}{def} \PY{n+nf}{test\PYZus{}error}\PY{p}{(}\PY{n+nb+bp}{self}\PY{p}{)}\PY{p}{:}
                  \PY{c+c1}{\PYZsh{} Test discriminant}
                  \PY{n+nb+bp}{self}\PY{o}{.}\PY{n}{assertRaises}\PY{p}{(}\PY{n+ne}{ValueError}\PY{p}{,} \PY{n}{calc\PYZus{}discriminant}\PY{p}{,} \PY{l+m+mi}{1}\PY{p}{,} \PY{l+m+mi}{0}\PY{p}{,} \PY{l+m+mi}{1}\PY{p}{)}
                  \PY{n+nb+bp}{self}\PY{o}{.}\PY{n}{assertRaises}\PY{p}{(}\PY{n+ne}{ValueError}\PY{p}{,} \PY{n}{calc\PYZus{}discriminant}\PY{p}{,} \PY{l+m+mi}{1}\PY{p}{,} \PY{l+m+mi}{1}\PY{p}{,} \PY{l+m+mi}{1}\PY{p}{)}
                  \PY{n+nb+bp}{self}\PY{o}{.}\PY{n}{assertRaises}\PY{p}{(}\PY{n+ne}{ValueError}\PY{p}{,} \PY{n}{calc\PYZus{}discriminant}\PY{p}{,} \PY{l+m+mi}{1}\PY{p}{,} \PY{o}{\PYZhy{}}\PY{l+m+mi}{1}\PY{p}{,} \PY{l+m+mi}{1}\PY{p}{)}
                  
                  \PY{c+c1}{\PYZsh{} Test sum}
                  \PY{n+nb+bp}{self}\PY{o}{.}\PY{n}{assertRaises}\PY{p}{(}\PY{n+ne}{ValueError}\PY{p}{,} \PY{n}{calc\PYZus{}sum}\PY{p}{,} \PY{o}{\PYZhy{}}\PY{l+m+mi}{3}\PY{p}{,} \PY{l+m+mi}{0}\PY{p}{)}
                  \PY{n+nb+bp}{self}\PY{o}{.}\PY{n}{assertRaises}\PY{p}{(}\PY{n+ne}{ValueError}\PY{p}{,} \PY{n}{calc\PYZus{}sum}\PY{p}{,} \PY{o}{\PYZhy{}}\PY{l+m+mi}{1}\PY{p}{,} \PY{o}{\PYZhy{}}\PY{l+m+mi}{1}\PY{p}{)}
                  \PY{n+nb+bp}{self}\PY{o}{.}\PY{n}{assertRaises}\PY{p}{(}\PY{n+ne}{ValueError}\PY{p}{,} \PY{n}{calc\PYZus{}sum}\PY{p}{,} \PY{o}{\PYZhy{}}\PY{l+m+mi}{2}\PY{p}{,} \PY{l+m+mi}{1}\PY{p}{)}
\end{Verbatim}


    \begin{Verbatim}[commandchars=\\\{\}]
{\color{incolor}In [{\color{incolor}258}]:} \PY{k}{if} \PY{n+nv+vm}{\PYZus{}\PYZus{}name\PYZus{}\PYZus{}} \PY{o}{==} \PY{l+s+s1}{\PYZsq{}}\PY{l+s+s1}{\PYZus{}\PYZus{}main\PYZus{}\PYZus{}}\PY{l+s+s1}{\PYZsq{}}\PY{p}{:}
              \PY{n}{unittest}\PY{o}{.}\PY{n}{main}\PY{p}{(}\PY{n}{argv}\PY{o}{=}\PY{p}{[}\PY{l+s+s1}{\PYZsq{}}\PY{l+s+s1}{first\PYZhy{}arg\PYZhy{}is\PYZhy{}ignored}\PY{l+s+s1}{\PYZsq{}}\PY{p}{]}\PY{p}{,} \PY{n}{exit}\PY{o}{=}\PY{k+kc}{False}\PY{p}{)}
\end{Verbatim}


    \begin{Verbatim}[commandchars=\\\{\}]
..
----------------------------------------------------------------------
Ran 2 tests in 0.002s

OK

    \end{Verbatim}


    % Add a bibliography block to the postdoc
    
    
    
    \end{document}
